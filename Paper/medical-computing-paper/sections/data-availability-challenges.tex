% !TEX root = ../main.tex

\section{Challenges in Obtaining Long-Term Real-World Data for Aflibercept}

\subsection{Limited Long-Term Studies}

While aflibercept has demonstrated robust efficacy in controlled clinical trials, obtaining long-term real-world data presents significant challenges. Unlike ranibizumab, which benefits from the SEVEN-UP study providing 7-year outcomes data \citep{martinRanibizumabBevacizumabTreatment2012}, aflibercept lacks comparable long-term follow-up studies. Most available aflibercept data extends only to 4-5 years post-treatment initiation, with the majority of studies reporting 2-3 year outcomes.

\subsection{High Attrition Rates}

Real-world studies consistently demonstrate substantial patient attrition over time. In a comprehensive analysis, follow-up data availability declined from 82.5\% at year 1 to only 15.0\% at year 5. This dramatic reduction in cohort size limits the statistical power and generalizability of long-term outcomes assessment. Natural mortality in the elderly AMD population accounts for a significant proportion of this attrition, but treatment discontinuation and loss to follow-up also contribute substantially.

\subsection{Treatment Discontinuation and Non-Persistence}

Systematic reviews indicate that approximately 50\% of patients discontinue anti-VEGF treatment by 24 months, with significant drop-off occurring within the first 6-12 months. For aflibercept specifically, real-world injection frequencies consistently fall below clinical trial protocols, with patients receiving an average of 7-8 injections in the first year compared to 8-9 in controlled trials. This undertreatment pattern complicates efforts to assess the drug's true long-term effectiveness.

\subsection{Data Collection Limitations}

Real-world data sources vary significantly in their design, consistency, and captured variables. Electronic medical record systems often lack standardized data collection protocols, leading to incomplete or inconsistent documentation of visual acuity measurements, injection intervals, and reasons for treatment discontinuation. Additionally, patients who see multiple providers or experience gaps in care may appear as treatment discontinuations in databases when they are actually continuing therapy elsewhere.

\subsection{Absence of Control Groups}

Unlike the ranibizumab trials that included sham-controlled arms providing natural history data, most aflibercept real-world studies lack untreated control groups. This absence makes it difficult to distinguish between treatment effects and natural disease progression, particularly when assessing long-term outcomes such as geographic atrophy development or vision loss despite continued therapy.

\subsection{Geographic and Healthcare System Variations}

The majority of long-term aflibercept data comes from specific healthcare systems or geographic regions, primarily Japan and select European countries. Healthcare delivery models, reimbursement structures, and treatment protocols vary substantially across regions, limiting the generalizability of findings. For instance, treat-and-extend protocols common in some regions may produce different outcomes than the fixed dosing regimens used elsewhere.

\subsection{Implications for Simulation Modeling}

These data limitations present significant challenges for developing accurate simulation models of aflibercept treatment outcomes. Key parameters such as long-term discontinuation rates, progression to geographic atrophy, and vision outcomes beyond 5 years must be extrapolated from limited data or inferred from ranibizumab studies. While the SEVEN-UP study provides valuable insights into the treated natural history of anti-VEGF therapy, the assumption that aflibercept follows similar long-term patterns remains unvalidated.

The paucity of long-term aflibercept data necessitates careful consideration of uncertainty in simulation parameters and highlights the need for sensitivity analyses to explore the impact of different assumptions on model outcomes. Future research priorities should include establishing registries for systematic long-term follow-up and standardizing data collection protocols across healthcare systems to better understand the real-world effectiveness of aflibercept therapy over extended time horizons.