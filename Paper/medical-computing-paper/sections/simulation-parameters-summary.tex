% !TEX root = ../main.tex

\section{Summary of Simulation Parameters from Literature}

\subsection{Core Treatment Parameters}

\subsubsection{Injection Frequency Distribution}
Based on the systematic review of long-term aflibercept studies, the following injection frequency parameters should be incorporated into the simulation model:

\begin{table}[h]
\centering
\caption{Annual Injection Frequencies for Aflibercept (Real-World Data)}
\begin{tabular}{lcc}
\hline
Year & Mean Injections & Range/SD \\
\hline
Year 1 & 7.0 & 6.8--7.7 \\
Year 2 & 3.3 & 2.5--4.0 \\
Years 3--5 & 3.2 & 2.7--3.8 \\
\hline
\end{tabular}
\end{table}

\paragraph{Critical Finding:} The bimodal distribution observed by \citet{nishikawaFourYearOutcomeAflibercept2019} suggests two distinct patient phenotypes:
\begin{itemize}
    \item Low-need cohort: Minimal injections after loading phase (approximately 30\% of patients)
    \item High-need cohort: Continuous regular injections throughout treatment period
\end{itemize}

\subsubsection{Treatment Thresholds}
\begin{itemize}
    \item \textbf{Minimum effective dose}: $\geq$5 injections annually to maintain visual gains \citep{kimImpactInjectionFrequency2020}
    \item \textbf{Cumulative dose effect}: Patients receiving $\geq$20 injections over 5 years show 8.0 letters better VA than those receiving $<$20 injections (p=0.001)
\end{itemize}

\subsection{Clinical Outcome Trajectories}

\subsubsection{Visual Acuity Evolution}
\begin{table}[h]
\centering
\caption{Mean Visual Acuity Change from Baseline (ETDRS Letters)}
\begin{tabular}{lccc}
\hline
Timepoint & Mean Change & 95\% CI & Source \\
\hline
Month 3 & +7.0 & -- & Loading phase response \\
Year 1 & +5.5 & +3 to +8 & Multiple studies \\
Year 2 & +2.0 & 0 to +4 & Decline begins \\
Year 3 & +0.5 & -2 to +3 & Near baseline \\
Year 4 & -1.0 & -3 to +1 & Below baseline \\
Year 5 & -2.9 & -5 to 0 & \citet{kimImpactInjectionFrequency2020} \\
\hline
\end{tabular}
\end{table}

\subsubsection{Treatment Response Categories}
Based on the literature, patients should be stratified into response categories:
\begin{enumerate}
    \item \textbf{Good responders} (30\%): Gain $\geq$10 letters, maintain with minimal treatment
    \item \textbf{Moderate responders} (50\%): Gain 0--9 letters, require continuous treatment
    \item \textbf{Poor responders} (20\%): Lose letters despite treatment
\end{enumerate}

\subsection{Discontinuation and Attrition Parameters}

\subsubsection{Cumulative Discontinuation Rates}
\begin{table}[h]
\centering
\caption{Cumulative Discontinuation Rates by Year}
\begin{tabular}{lcc}
\hline
Timepoint & Discontinuation Rate & Primary Reasons \\
\hline
Year 1 & 10--15\% & Early non-response \\
Year 2 & 25--30\% & Treatment burden \\
Year 3 & 35--43\% & Multiple factors \\
Year 4 & 40--45\% & Disease stability/futility \\
Year 5 & 45--50\% & Cumulative attrition \\
\hline
\end{tabular}
\end{table}

\subsubsection{Discontinuation Categories}
\begin{itemize}
    \item \textbf{Planned discontinuation} (30\%): Disease stability, good visual outcome
    \item \textbf{Administrative discontinuation} (40\%): Lost to follow-up, patient choice
    \item \textbf{Medical discontinuation} (30\%): Treatment futility, adverse events, death
\end{itemize}

\subsection{Disease Progression Parameters}

\subsubsection{Complication Development Rates}
\begin{table}[h]
\centering
\caption{Cumulative Rates of Disease Complications}
\begin{tabular}{lcccc}
\hline
Complication & Year 1 & Year 3 & Year 5 & Year 10* \\
\hline
Geographic atrophy & 5\% & 12\% & 20\% & 49\% \\
Fibrosis/scarring & 3\% & 8\% & 15\% & 25\% \\
Subretinal hemorrhage & 2\% & 5\% & 8\% & 12\% \\
\hline
\multicolumn{5}{l}{*Extrapolated from ranibizumab data} \\
\end{tabular}
\end{table}

\subsection{Real-World vs Clinical Trial Gap}

The literature consistently demonstrates a substantial efficacy-effectiveness gap:
\begin{itemize}
    \item \textbf{Clinical trials}: +8--10 letters at 1 year, maintained at 2 years
    \item \textbf{Real-world}: +1--3 letters at 1 year, declining thereafter
    \item \textbf{Key driver}: Undertreatment (7.3 vs 10+ injections in year 1)
\end{itemize}

\subsection{Simulation Model Recommendations}

\subsubsection{Essential Parameters to Model}
\begin{enumerate}
    \item \textbf{Patient heterogeneity}: Implement distinct phenotypes based on treatment need
    \item \textbf{Treatment intensity}: Model as primary driver of outcomes
    \item \textbf{Time-dependent effects}: Incorporate declining treatment effectiveness
    \item \textbf{Competing risks}: Include mortality, geographic atrophy, and fibrosis
\end{enumerate}

\subsubsection{Uncertainty Quantification}
Given data limitations beyond 5 years, the following parameters require sensitivity analysis:
\begin{itemize}
    \item Long-term visual acuity trajectory (years 6--10)
    \item Geographic atrophy progression rate with aflibercept
    \item Impact of biosimilar adoption on treatment patterns
    \item Treatment discontinuation rates in stable patients
\end{itemize}

\subsubsection{Validation Targets}
The simulation should reproduce:
\begin{itemize}
    \item Year 1 injection frequency: 7.0 $\pm$ 0.7
    \item Year 1 VA gain: +5.5 $\pm$ 2.5 letters
    \item 5-year retention rate: 50--60\%
    \item 5-year mean VA change: -3 to 0 letters
\end{itemize}

These parameters provide a quantitative framework for implementing realistic aflibercept treatment patterns in the ABS simulation engine, with clear targets for calibration and validation.