\section{Discovery of Inappropriate Clinical Discontinuation Patterns}

\subsection{Background}

During analysis of treatment gaps in the SASH aflibercept cohort, we found that not all treatment disruptions were attributable to external factors such as the COVID-19 pandemic. A subset showed patterns suggesting misunderstandings about the chronic nature of neovascular AMD.

\subsection{Methods}

We analyzed 21,727 injection intervals from patients receiving aflibercept 2mg between 2015-2023. Premature discontinuations were defined as:

\begin{itemize}
    \item Visual acuity >20 letters at discontinuation
    \item Treatment interval increasing from $\leq$60 days to $\geq$180 days
    \item Exclusion of planned discontinuations at the one-year mark
\end{itemize}

Discontinuation patterns were categorized based on visual acuity levels and timing to identify potential clinical reasoning errors.

\subsection{Results}

\subsubsection{Overall Discontinuation Patterns}

Of 21,727 intervals analyzed:
\begin{itemize}
    \item 282 (1.3\%) represented inappropriate clinical discontinuations
    \item 264 unique patients were affected
    \item Mean visual acuity at discontinuation: 62.3 letters (Snellen 20/50)
    \item 95.4\% eventually restarted treatment
    \item Mean time to restart: 92 days
\end{itemize}

\subsubsection{Categories of Clinical Misunderstanding}

Analysis revealed four distinct patterns of inappropriate discontinuation:

\begin{table}[h]
\centering
\begin{tabular}{p{3.5cm}ccc}
\hline
\textbf{Category} & \textbf{n (\%)} & \textbf{Mean VA} & \textbf{VA Loss} \\
\hline
"Too good to stop" & 89 (31.6\%) & 76.3 letters & -12.6 letters \\
"Course complete" & 40 (14.2\%) & 63.8 letters & -14.2 letters \\
"Good enough" & 52 (18.4\%) & 63.6 letters & -7.3 letters \\
"Plateau reasoning" & 44 (15.6\%) & 43.7 letters & -6.8 letters \\
Other & 57 (20.2\%) & 52.8 letters & -6.6 letters \\
\hline
\end{tabular}
\caption{Clinical discontinuation patterns and outcomes}
\end{table}

\subsubsection{Visual Acuity and Discontinuation Risk}

Patients with the best visual outcomes were at highest risk of inappropriate discontinuation. The "too good to stop" group:

\begin{itemize}
    \item Had excellent vision (mean 76.3 letters, Snellen 20/25)
    \item Comprised the largest category (31.6\%)
    \item Experienced the greatest vision loss (-12.6 letters)
    \item Suggested misunderstanding of AMD as curable
\end{itemize}

\subsubsection{Time-Based Discontinuation Patterns}

Forty patients (14.2\%) were discontinued around the one-year mark with good vision, suggesting clinicians viewed aflibercept therapy like an antibiotic course rather than chronic disease management. This group experienced the largest vision loss with mean decline of 14.2 letters.

\subsubsection{Comparison with External Gaps}

Inappropriate clinical discontinuations showed worse outcomes than COVID-related gaps:

\begin{table}[h]
\centering
\begin{tabular}{lcc}
\hline
\textbf{Gap Type} & \textbf{Prevalence} & \textbf{Mean VA Loss} \\
\hline
COVID gaps (3-6 months) & 9.4\% & -5.9 letters \\
COVID gaps (6-12 months) & 2.5\% & -8.1 letters \\
Clinical errors (all) & 1.3\% & -9.5 letters \\
\hline
\end{tabular}
\caption{Comparison of gap types and visual outcomes}
\end{table}

\subsection{Clinical Implications}

\subsubsection{Education Priorities}

These findings suggest areas for clinical education:

\begin{enumerate}
    \item \textbf{AMD chronicity}: Neovascular AMD requires lifelong management
    \item \textbf{No cure concept}: Good visual outcomes reflect treatment success, not disease cure
    \item \textbf{Maintenance paradigm}: Excellent vision requires continued treatment
    \item \textbf{Plateau interpretation}: Stable vision indicates treatment efficacy, not futility
\end{enumerate}

\subsubsection{Risk Stratification}

Patients at highest risk for inappropriate discontinuation:
\begin{itemize}
    \item Those achieving excellent visual outcomes (>70 letters)
    \item Patients completing approximately one year of treatment
    \item Those with good but not excellent vision (50-70 letters) early in treatment
\end{itemize}

\subsection{Modeling Implications}

To accurately simulate real-world outcomes, we developed a dual-protocol approach:

\begin{enumerate}
    \item \textbf{Standard protocol}: Representing best practice without clinical errors
    \item \textbf{SASH protocol}: Including the observed 1.3\% inappropriate discontinuation rate
\end{enumerate}

This allows quantification of the "cost of misunderstanding" – the gap between achievable and actual outcomes due to clinical decision errors.

\subsection{Limitations}

\begin{itemize}
    \item Single-center data may not represent universal patterns
    \item Retrospective categorization of clinical reasoning
    \item Unable to capture clinician-patient discussions
    \item Temporal trends in practice patterns not analyzed
\end{itemize}

\subsection{Conclusions}

This analysis shows that real-world AMD outcomes depend on both therapeutic effectiveness and clinical understanding of disease chronicity. The finding that patients with the best visual outcomes face the highest risk of inappropriate discontinuation warrants attention. The 1.3\% rate of clinical decision errors, while seemingly small, disproportionately affects those who could maintain excellent vision with continued treatment.

These findings underscore the importance of:
\begin{itemize}
    \item Continuous medical education emphasizing AMD chronicity
    \item Clear guidelines discouraging discontinuation based on good outcomes
    \item Patient education about the need for lifelong treatment
    \item Regular audit of discontinuation patterns
\end{itemize}

The implications of these preventable vision losses suggest value in clinical education alongside therapeutic development.