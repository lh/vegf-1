\section{Methods: COVID-Era Gap Analysis}

\subsection{Data Source and Study Period}

We analyzed a comprehensive dataset of aflibercept 2mg injections from the St George's and St Helier's (SASH) ophthalmology service covering the period 2015-2023. This timeframe encompassed both pre-pandemic practice patterns and the COVID-19 disruption period, allowing comparison of routine and disrupted care.

\subsection{Injection Interval Analysis}

\subsubsection{Data Structure}
The dataset comprised 21,727 inter-injection intervals from patients receiving aflibercept 2mg for neovascular AMD. Each interval record included:
\begin{itemize}
    \item Previous injection date and visual acuity
    \item Current injection date and visual acuity
    \item Calculated interval duration (days)
    \item Patient and eye identifiers
\end{itemize}

\subsubsection{Gap Categorization}
Intervals were categorized based on duration to reflect clinical significance:
\begin{itemize}
    \item \textbf{Regular treatment}: $\leq$90 days (reflecting monthly to 3-monthly planned intervals)
    \item \textbf{Short gaps}: 91-180 days (3-6 months, potentially recoverable)
    \item \textbf{Long gaps}: 181-365 days (6-12 months, significant disruption)
    \item \textbf{Effective discontinuation}: >365 days
\end{itemize}

\subsection{Visual Acuity Impact Assessment}

For each interval, we calculated:
\begin{equation}
\text{VA change} = \text{VA}_{\text{current}} - \text{VA}_{\text{previous}}
\end{equation}

Monthly vision loss rates during gaps were estimated as:
\begin{equation}
\text{Monthly VA loss} = \frac{\text{VA change}}{\text{Interval days}} \times 30
\end{equation}

\subsection{Recovery Pattern Analysis}

For patients experiencing gaps, we tracked subsequent visits to assess recovery:
\begin{enumerate}
    \item Identified all gaps >90 days
    \item Tracked visual acuity at the visit following the gap
    \item Compared to pre-gap baseline
    \item Calculated recovery rate and magnitude
\end{enumerate}

\subsection{Clinical Decision Analysis}

To differentiate between external disruptions and clinical decisions, we applied specific criteria for "premature discontinuation":
\begin{itemize}
    \item Visual acuity >20 letters at discontinuation
    \item Interval increase from $\leq$60 days to $\geq$180 days
    \item Exclusion of discontinuations at approximately one year (330-390 days from treatment initiation)
\end{itemize}

\subsection{Statistical Methods}

\subsubsection{Descriptive Statistics}
We calculated means, medians, and standard deviations for:
\begin{itemize}
    \item Interval durations by category
    \item Visual acuity changes
    \item Time to treatment restart
\end{itemize}

\subsubsection{Risk Quantification}
We computed the probability of significant vision loss (5, 10, and 15 letter thresholds) as a function of gap duration using logistic regression.

\subsubsection{Time-Dependent Analysis}
Vision loss rates were analyzed in time bins to identify non-linear patterns:
\begin{itemize}
    \item 0-3 months
    \item 3-6 months  
    \item 6-12 months
    \item 1-2 years
    \item >2 years
\end{itemize}

\subsection{Clinical Reasoning Categorization}

For identified premature discontinuations, we developed a classification system based on:
\begin{itemize}
    \item Visual acuity at discontinuation
    \item Time since treatment initiation
    \item Subsequent clinical course
\end{itemize}

Categories were validated through:
\begin{enumerate}
    \item Review of aggregated patterns
    \item Clinical plausibility assessment
    \item Outcome consistency within categories
\end{enumerate}

\subsection{Ethical Considerations}

This analysis used fully anonymized retrospective data with no patient identifiers. The study aimed to improve care quality through identification of modifiable practice patterns.