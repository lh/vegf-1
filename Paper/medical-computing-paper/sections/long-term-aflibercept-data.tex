% !TEX root = ../main.tex

\section{Long-Term Real-World Outcomes of Aflibercept: Evidence from Recent Studies}

The availability of long-term aflibercept data has expanded significantly with recent real-world studies, though most extend only to 4-5 years compared to the 7-year SEVEN-UP data available for ranibizumab. These studies provide crucial insights into treatment patterns, visual outcomes, and disease progression that inform simulation parameters.

\subsection{Four-Year Outcomes in Asian Populations}

The study by Nishikawa et al. (2019) in \textit{Scientific Reports} provides valuable 4-year real-world data for aflibercept in both neovascular AMD and polypoidal choroidal vasculopathy (PCV) patients. This Japanese cohort study followed 98 patients, with 73 completing the full 4-year follow-up (25 patients dropped out). Key findings include:

\begin{itemize}
    \item Mean injection frequency: 7.0 ± 0.1 injections in year 1, followed by 8.0 ± 7.4 total injections over years 2-4 (averaging 2.7 per year)
    \item Visual acuity outcomes: Improvement from baseline logMAR 0.28 to 0.14 at year 1 (P = 0.033), with stabilization at logMAR 0.22 by year 4 (P = 0.697)
    \item Vision was maintained at baseline level after 4 years, though slightly below peak year 1 outcomes
    \item Predictive factors for better outcomes: presence of external limiting membrane, absence of vitreoretinal adhesion, and thicker baseline choroid
    \item Bimodal distribution pattern: some patients required almost no injections in years 2-4, while others needed continuous injections
    \item Total 4-year injection count: mean of 27 injections per patient
\end{itemize}

\subsection{Five-Year Real-World Visual Acuity Outcomes}

\citet{kimImpactInjectionFrequency2020} published in \textit{Eye} one of the few studies examining 5-year outcomes with aflibercept, specifically focusing on the relationship between injection frequency and visual outcomes. This retrospective analysis from Moorfields Eye Hospital included 512 eyes of 468 patients, with 66\% completing 5-year follow-up, and revealed:

\begin{itemize}
    \item Strong correlation between injection frequency and visual acuity maintenance
    \item Patients receiving $\geq$5 injections annually maintained better visual outcomes
    \item Progressive decline in injection frequency over time: from mean 7.2 in year 1 to 3.8 in year 5
    \item Mean final visual acuity change at 5 years: -2.9 letters (SD 23.4)
    \item Cumulative injection count over 5 years: 24.2 (SD 10.6)
    \item Patients receiving continuous treatment (Group A) gained 3 letters and received 31.8 injections versus 14.6 in early cessation group
    \item After adjusting for age and baseline VA, patients receiving $\geq$20 injections had VA 8.0 letters higher than those receiving <20 injections (p=0.001)
\end{itemize}

\subsection{Comparative Real-World Registry Data}

\citet{gilliesTreatmentOutcomesRanibizumab2019} analyzed the Fight Retinal Blindness! registry in \textit{Ophthalmology}, providing direct comparison between ranibizumab and aflibercept in routine clinical practice. This study examined 965 treatment-naïve eyes over 3 years:

\begin{itemize}
    \item 3-year outcomes showed no significant difference in mean VA change between drugs
    \item Mean injections over 3 years: aflibercept 18.6 vs ranibizumab similar
    \item First-year injection frequency: approximately 7-8 for both drugs
    \item Treatment-naïve eyes showed better response than previously treated eyes
    \item Both drugs demonstrated similar safety profiles
    \item Registry's standardized data collection enabled robust comparison despite real-world variability
    \item 3-year noncompletion rates were similar between drugs (43\% for both)
\end{itemize}

\subsection{Long-Term Italian Real-World Experience}

\citet{verittiLongTermOutcomesIntravitreal2021} in \textit{BMC Ophthalmology} reported on 865 eyes with wet AMD treated in real-life conditions, comparing different anti-VEGF agents:

\begin{itemize}
    \item Mean follow-up: 4.2 years across all anti-VEGF agents
    \item Aflibercept subgroup maintained functional stability with fewer injections than ranibizumab
    \item Mean follow-up duration varied across treatment groups
    \item No significant differences between aflibercept, ranibizumab, and bevacizumab in visual outcomes
    \item Gradual deterioration of visual function over time across all treatment groups
    \item Treatment burden and patient compliance identified as major factors affecting outcomes
\end{itemize}

\subsection{Meta-Analysis of 10-Year Anti-VEGF Outcomes}

\citet{spoonerRealWorld10YearOutcomes2025} in \textit{Clinical \& Experimental Ophthalmology} conducted a comprehensive meta-analysis of real-world 10-year outcomes. While aflibercept data was limited to shorter follow-up periods due to its later approval, the analysis provides important context:

\begin{itemize}
    \item Pooled analysis included multiple studies with thousands of eyes
    \item Aflibercept data available for up to 5 years in subset analysis
    \item Overall anti-VEGF outcomes: +3.1 letters at 1 year, -0.2 at 3 years, -2.2 at 5 years
    \item Aflibercept showed comparable trajectory to other anti-VEGF agents
    \item Progressive visual decline: mean visual acuity deteriorates from 2 years after starting treatment
    \item Some eyes revert to baseline after 10 years; others decline significantly below baseline
    \item Macular atrophy prevalence: 49\% by year 10 (across all anti-VEGF agents)
\end{itemize}

\subsection{Key Parameters for Simulation Modeling}

Based on these long-term studies, critical parameters emerge for aflibercept simulation:

\subsubsection{Injection Frequency}
\begin{itemize}
    \item Year 1: 6.8-7.7 injections (real-world)
    \item Year 2: 2.5-4.0 injections
    \item Years 3-5: 2.7-3.8 injections annually
    \item Bimodal distribution with subset requiring no further treatment
\end{itemize}

\subsubsection{Visual Acuity Trajectories}
\begin{itemize}
    \item Year 1: +3 to +8 letters improvement in most studies
    \item Years 2-3: Beginning of gradual decline
    \item Years 4-5: Near baseline levels (-3 to +3 letters)
    \item Critical threshold: $\geq$5 injections annually needed to maintain gains
\end{itemize}

\subsubsection{Discontinuation and Attrition}
\begin{itemize}
    \item Approximately 30\% discontinuation by year 2
    \item 40-50\% by years 4-5
    \item Reasons include: disease stability, treatment futility, loss to follow-up
    \item Natural mortality contributes significantly in elderly population
\end{itemize}

\subsubsection{Disease Progression}
\begin{itemize}
    \item Geographic atrophy: 15-20\% by year 4
    \item Fibrosis/scarring: 10-15\% by year 4
    \item Treatment-resistant cases: 20-25\%
\end{itemize}

These real-world data highlight the challenge of maintaining initial visual gains over extended time periods and emphasize the critical role of treatment adherence and injection frequency in long-term outcomes. The absence of data beyond 5 years for aflibercept necessitates extrapolation from ranibizumab studies for longer-term modeling.