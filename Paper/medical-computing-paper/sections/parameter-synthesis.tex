\section{Parameter Synthesis from Multiple Evidence Sources}

\subsection{Overview}

The development of simulation parameters for aflibercept 2mg in age-related macular degeneration (AMD) required synthesis of evidence from multiple sources including randomized controlled trials (RCTs), real-world registries, and local clinical practice data. This approach highlighted differences between clinical trial efficacy and real-world effectiveness.

\subsection{Evidence Hierarchy and Data Sources}

Our parameter extraction followed a structured hierarchy of evidence:

\begin{enumerate}
    \item \textbf{Pivotal RCTs} (n=1,464): VIEW 1 and VIEW 2 trials provided foundational efficacy data
    \item \textbf{Treat-and-extend RCTs} (n=288): ALTAIR and smaller studies (HAGA, Maruko) defined extension protocols
    \item \textbf{Real-world registries} (n=3,313): Fight Retinal Blindness! (FRB!) registry quantified the efficacy-effectiveness gap
    \item \textbf{Discontinuation studies} (n=102): Aslanis et al. characterized recurrence patterns
    \item \textbf{Local practice data} (n=21,727 intervals): SASH aflibercept cohort revealed treatment disruption patterns
\end{enumerate}

\subsection{Visual Acuity Outcomes: RCT versus Real-World}

The synthesis showed reduced effectiveness in real-world settings:

\begin{table}[h]
\centering
\begin{tabular}{lcc}
\hline
\textbf{Study Type} & \textbf{2-Year VA Gain} & \textbf{Injection Count} \\
\hline
VIEW 1/2 (RCT) & +7.6 letters & 11.2 \\
ALTAIR (RCT) & +6.1--7.6 letters & 10.4 \\
FRB! (Real-world) & +4.2 letters & 14.9 \\
\hline
\end{tabular}
\caption{Efficacy-effectiveness gap in aflibercept treatment}
\end{table}

This represents a real-world effectiveness of approximately 55-70\% of RCT efficacy.

\subsection{Disease State Transition Modeling}

We synthesized transition probabilities from ALTAIR's treat-and-extend data, defining four disease states:

\begin{itemize}
    \item \textbf{NAIVE}: Pre-treatment state
    \item \textbf{STABLE}: Achieving interval extension (59\% post-loading)
    \item \textbf{ACTIVE}: Maintaining current interval (33\% post-loading)
    \item \textbf{HIGHLY\_ACTIVE}: Requiring interval reduction (8\% post-loading)
\end{itemize}

Notably, 80-85\% of patients who achieve STABLE state maintain it per decision interval.

\subsection{Treatment Protocol Variations}

Analysis of multiple treat-and-extend protocols showed flexibility in implementation:

\begin{itemize}
    \item \textbf{Maximum interval}: 16 weeks (ALTAIR) vs 12 weeks (Maruko/HAGA)
    \item \textbf{Extension increment}: 2 weeks (ALTAIR) vs 4 weeks (Maruko)
    \item \textbf{Outcome similarity}: Despite protocol differences, 2-year outcomes were comparable
\end{itemize}

This suggests that the specific protocol details may be less critical than consistent application and monitoring.

\subsection{Gap Consequence Quantification}

Analysis of 21,727 injection intervals during the COVID-19 era quantified treatment gap consequences:

\begin{table}[h]
\centering
\begin{tabular}{lccc}
\hline
\textbf{Gap Duration} & \textbf{Prevalence} & \textbf{VA Loss Rate} & \textbf{Net Impact} \\
\hline
3-6 months & 9.4\% & 0.47 letters/month & -5.9 letters \\
6-12 months & 2.5\% & 0.81 letters/month & -8.1 letters \\
>12 months & 1.6\% & 0.51 letters/month & -11.8 letters \\
\hline
\end{tabular}
\caption{Vision loss during treatment gaps}
\end{table}

Vision loss rate peaked at 6-12 months then plateaued, possibly reflecting selection effects in patients tolerating longer gaps.

\subsection{Recovery Patterns Post-Gap}

Analysis revealed partial recovery potential after treatment resumption:

\begin{itemize}
    \item 50.5\% of patients showed recovery after 3-6 month gaps
    \item 55.4\% showed recovery after 6-12 month gaps
    \item Mean recovery was 1-2 letters, leaving net deficits of 6-8 letters
\end{itemize}

This contrasts with Aslanis et al.'s findings of full recovery after planned discontinuation with monitoring.

\subsection{Population-Specific Considerations}

Meta-analysis showed population differences:

\begin{itemize}
    \item \textbf{Polypoidal choroidal vasculopathy (PCV)}: 36-76\% in Japanese studies vs ~10\% Western
    \item \textbf{Treatment response}: Higher gains in PCV-predominant cohorts (HAGA: +13-16 letters)
    \item \textbf{Extension achievement}: Cultural and system factors affected maximum intervals achieved
\end{itemize}

\subsection{Integrated Parameter Model}

The final parameter synthesis incorporated:

\begin{enumerate}
    \item \textbf{Base efficacy}: VIEW trial outcomes
    \item \textbf{Transition dynamics}: ALTAIR treat-and-extend patterns
    \item \textbf{Real-world adjustment}: 0.6-0.7 effectiveness factor from FRB!
    \item \textbf{Gap consequences}: Time-dependent vision loss from COVID-era analysis
    \item \textbf{Recovery patterns}: Partial recovery modeling from local data
\end{enumerate}

\subsection{Validation Approach}

Parameters were validated through:

\begin{itemize}
    \item Cross-checking injection frequencies across studies
    \item Comparing predicted vs observed vision outcomes
    \item Sensitivity analysis on key parameters
    \item Clinical expert review for face validity
\end{itemize}

\subsection{Implications for Simulation}

\begin{enumerate}
    \item \textbf{Real-world adjustment is essential}: RCT parameters alone overestimate outcomes by 30-45\%
    \item \textbf{Gap consequences are severe but predictable}: Vision loss follows consistent patterns
    \item \textbf{Protocol flexibility exists}: Similar outcomes achievable with different extension schemes
    \item \textbf{Population matters}: Parameters may need adjustment for specific demographics
    \item \textbf{Recovery is limited}: Treatment gaps cause partially irreversible damage
\end{enumerate}

This parameter synthesis provides a foundation for AMD treatment simulation that accounts for differences between clinical trial and real-world outcomes.