\section{Parameter Extraction and Simulation Framework}

\subsection{Methodology for Parameter Extraction}

To develop a robust simulation of aflibercept treatment for nAMD, we systematically extracted key parameters from clinical trials and real-world evidence studies. Our approach involved standardizing definitions across studies and reconciling differences between controlled trial outcomes and real-world observations.

\subsection{Disease State Definitions}

For simulation purposes, we standardized disease states across studies as follows:
\begin{itemize}
    \item \textbf{NAIVE:} Patients before first injection
    \item \textbf{STABLE:} Patients with interval extension or maintaining maximum interval
    \item \textbf{ACTIVE:} Patients maintaining their current interval (except maximum interval)
    \item \textbf{HIGHLY\_ACTIVE:} Patients requiring treatment interval reduction
\end{itemize}

\subsection{Key Parameters for Simulation}

Table~\ref{tab:simulation_parameters} presents the consolidated parameters extracted from the literature review, categorized by parameter type and showing the source studies. These parameters form the foundation of our agent-based and discrete event simulation models.

\begin{table}[h]
\begin{threeparttable}
\caption{Key Parameters for AMD Treatment Simulation}
\label{tab:simulation_parameters}
\begin{tabular}{p{4.5cm}p{5cm}p{3.5cm}p{2cm}}
\hline
\textbf{Parameter} & \textbf{Value} & \textbf{Source} & \textbf{Confidence} \\
\hline
\multicolumn{4}{l}{\textit{Visual Acuity Parameters}} \\
\hline
NAIVE treatment response & +8.4 letters (SD 1.2-1.4) & VIEW 1/2\tnote{1}, ALTAIR\tnote{2} & High \\
STABLE treatment response & +1.0 to +2.0 letters (SD 0.5-1.0) & ALTAIR\tnote{2} & Medium \\
ACTIVE treatment response & +0.5 to +1.0 letters (SD 0.5-1.0) & ALTAIR\tnote{2} & Medium \\
HIGHLY\_ACTIVE response & -0.5 to +1.0 letters (SD 1.0-1.5) & ALTAIR\tnote{2} & Medium \\
Real-world mean VA change & +1.0 letter (at 1 year) & Ciulla et al.\tnote{3} & High \\
\hline
\multicolumn{4}{l}{\textit{Disease State Transition Probabilities}} \\
\hline
NAIVE to STABLE & 0.55-0.60 & ALTAIR\tnote{2} & High \\
NAIVE to ACTIVE & 0.30-0.35 & ALTAIR\tnote{2} & Medium \\
NAIVE to HIGHLY\_ACTIVE & 0.05-0.10 & ALTAIR\tnote{2} & Medium \\
STABLE persistence & 0.80-0.85 & ALTAIR\tnote{2} & Medium \\
ACTIVE persistence & 0.55-0.60 & ALTAIR\tnote{2} & Medium \\
HIGHLY\_ACTIVE persistence & 0.50-0.65 & ALTAIR\tnote{2} & Low \\
\hline
\multicolumn{4}{l}{\textit{Treatment Protocol Parameters}} \\
\hline
Loading phase injections & 3 & All studies & High \\
Initial interval & 8 weeks & All studies & High \\
Minimum interval & 8 weeks & ALTAIR\tnote{2} & High \\
Maximum interval & 16 weeks & ALTAIR\tnote{2} & High \\
Extension increment & 2 or 4 weeks & ALTAIR\tnote{2} & High \\
Mean injections (year 1) & 7.3 (real-world) & Ciulla et al.\tnote{3} & High \\
Mean injections (2 years) & 10.4 (clinical trial) & ALTAIR\tnote{2} & High \\
\hline
\multicolumn{4}{l}{\textit{Treatment Effect by Injection Frequency}} \\
\hline
≤4 injections/year & -2 to -3 letters & Ciulla et al.\tnote{3} & High \\
5-6 injections/year & -0.4 to -1.6 letters & Ciulla et al.\tnote{3} & High \\
7-8 injections/year & +0.7 to +2.1 letters & Ciulla et al.\tnote{3} & High \\
9-10 injections/year & +2.4 to +3.3 letters & Ciulla et al.\tnote{3} & High \\
11-13 injections/year & +3.0 to +4.3 letters & Ciulla et al.\tnote{3} & High \\
\hline
\multicolumn{4}{l}{\textit{Treatment Effect by Baseline VA}} \\
\hline
≤20/200 & +13.9 letters & Ciulla et al.\tnote{3} & High \\
20/70-20/200 & +0.8 letters & Ciulla et al.\tnote{3} & High \\
20/40-20/70 & -0.8 letters & Ciulla et al.\tnote{3} & High \\
≥20/40 & -3.3 letters & Ciulla et al.\tnote{3} & High \\
\hline
\multicolumn{4}{l}{\textit{Discontinuation and Recurrence Parameters}} \\
\hline
Recurrence rate at 12 months & 52.9\% & Aslanis et al.\tnote{4} & High \\
Mean time to recurrence & 6.7 ± 2.2 months & Aslanis et al.\tnote{4} & High \\
4-month recurrence rate & 13\% & Aslanis et al.\tnote{4} & High \\
6-month recurrence rate & 33\% & Aslanis et al.\tnote{4} & High \\
8-month recurrence rate & 46\% & Aslanis et al.\tnote{4} & High \\
VA change at recurrence & -3.6 letters & Aslanis et al.\tnote{4} & High \\
VA after re-treatment & -0.3 letters (from baseline) & Aslanis et al.\tnote{4} & High \\
PED recurrence risk & 74\% (vs. 48\% without PED) & Aslanis et al.\tnote{4} & Medium \\
\hline
\end{tabular}
\begin{tablenotes}
\small
\item[1] Heier et al. (2012) VIEW 1/2 study \citep{heierIntravitrealAfliberceptVEGF2012}
\item[2] Ohji et al. (2020) ALTAIR study \citep{ohjiEfficacySafetyIntravitreal2020}
\item[3] Ciulla et al. (2020) 49,485-eye real-world study \citep{ciullaVisualAcuityOutcomes2020}
\item[4] Aslanis et al. (2022) treatment discontinuation study \citep{aslanisRecurrentNeovascularAgeRelated2022}
\end{tablenotes}
\end{threeparttable}
\end{table}

\subsection{Parameter Integration Approach}

To reconcile differences between clinical trial outcomes and real-world evidence, we developed a composite approach:

\begin{itemize}
    \item \textbf{Clinical Trial Baseline}: Visual acuity parameters and disease state transitions were primarily derived from the ALTAIR study \citep{ohjiEfficacySafetyIntravitreal2020}, which provided detailed treat-and-extend outcomes with aflibercept.
    
    \item \textbf{Real-World Adjustment}: We applied scaling factors based on Ciulla et al. \citep{ciullaVisualAcuityOutcomes2020} to adjust the clinical trial visual outcomes to real-world expectations, accounting for differences in treatment adherence, monitoring frequency, and patient selection.
    
    \item \textbf{Treatment Intensity Modifiers}: The linear relationship between injection frequency and visual outcomes identified by Ciulla et al. \citep{ciullaVisualAcuityOutcomes2020} was incorporated as a modifier function.
    
    \item \textbf{Baseline Vision Ceiling Effects}: Special adjustment functions were implemented to account for the observed ceiling effects in patients with good baseline vision.
\end{itemize}

\subsection{Simulation Implementation}

The parameters were implemented in both agent-based and discrete event simulation frameworks to model:

\begin{enumerate}
    \item Individual patient trajectories based on treatment decisions
    \item Visual acuity changes over time
    \item Disease state transitions
    \item Resource utilization (clinic visits, injections)
    \item Treatment discontinuation and potential recurrence
\end{enumerate}

The simulation incorporated three key time points from the AMD Treatment Three-Point Trajectory Model:

\begin{enumerate}
    \item \textbf{Baseline VA} ($T_0$): Starting visual acuity
    \item \textbf{Maximum VA} ($T_{max}$): Peak visual acuity after loading phase
    \item \textbf{1-Year VA} ($T_{12}$): Visual acuity at 12 months
\end{enumerate}

For $T_{max}$, we calculated:

\begin{equation*}
\text{Max\_VA} = \text{Baseline\_VA} + (\text{Base\_Response} \times \text{Disease\_State\_Modifier} \times \text{Baseline\_VA\_Modifier})
\end{equation*}

For $T_{12}$, we implemented a combined approach:

\begin{equation*}
\text{Year1\_VA} = \text{Max\_VA} + (0.6 \times \text{Intensity\_Based\_Change}) + (0.4 \times \text{Activity\_Based\_Change} \times \text{Adherence\_Factor})
\end{equation*}

This framework allows for comprehensive modeling of aflibercept treatment outcomes, balancing clinical trial efficacy with real-world effectiveness factors.
