% !TEX root = ../main.tex

\section{Literature Review on Aflibercept Treatment for Neovascular AMD}

Neovascular age-related macular degeneration (nAMD) represents a significant cause of vision loss in older populations. The introduction of anti-vascular endothelial growth factor (anti-VEGF) therapy has revolutionized treatment approaches, with aflibercept emerging as an effective therapeutic option.

\subsection{Clinical Trial Evidence}

The efficacy of aflibercept 2mg was initially established in the pivotal VIEW 1 and VIEW 2 trials, which demonstrated non-inferiority to monthly ranibizumab with bimonthly dosing after three initial monthly injections \citep{heierIntravitrealAfliberceptVEGF2012, schmidt-erfurthIntravitrealAfliberceptInjection2014}. These studies showed a mean improvement of 8.4 letters at 52 weeks, with approximately 31\% of patients gaining 15 or more letters \citep{heierIntravitrealAfliberceptVEGF2012}.

Building on these findings, the ALTAIR study examined treat-and-extend (T\&E) regimens with different adjustment intervals (2-week vs. 4-week) in Japanese patients. After 96 weeks, both groups maintained similar visual gains (+6.1 to +7.6 letters) while reducing treatment burden to approximately 10.4 injections over two years \citep{ohjiEfficacySafetyIntravitreal2020}. Notably, 41.5-46.3\% of patients achieved a 16-week treatment interval by week 96, suggesting significant durability potential for aflibercept in many patients.

Similar outcomes were observed in another Japanese T\&E study by Maruko et al., though with a more conservative extension approach limiting intervals to 12 weeks maximum, resulting in slightly higher injection frequency (13.0 injections over two years) \citep{marukoTwoYearOutcomesTreatandExtend2020}.

\subsection{Real-World Effectiveness}

Despite promising clinical trial results, real-world evidence suggests more modest outcomes. In a comprehensive analysis of 49,485 eyes in the United States, Ciulla et al. found that patients received a mean of 7.3 injections in the first year but achieved only a 1-letter mean improvement \citep{ciullaVisualAcuityOutcomes2020}. Importantly, this study revealed a linear relationship between injection frequency and visual gains, with better outcomes observed in patients receiving 9 or more injections annually.

The gap between clinical trials and real-world outcomes highlights challenges in treatment implementation, including undertreatment and variable adherence to recommended protocols. Baseline vision also significantly impacts outcomes, with patients having worse initial vision ($\leq$20/200) gaining substantially more letters than those with better baseline vision ($\geq$20/40), who tend to experience slight vision loss despite treatment \citep{ciullaVisualAcuityOutcomes2020}.

\subsection{Treatment Discontinuation}

An important clinical question concerns the possibility of discontinuing treatment after disease stabilization. Aslanis et al. investigated this in patients who had shown disease stability through three consecutive 12-week treatment intervals. Their prospective study revealed that 52.9\% of patients experienced disease recurrence within 12 months after treatment cessation, with a mean time to recurrence of 6.7 months \citep{aslanisRecurrentNeovascularAgeRelated2022}. Notably, the presence of pigment epithelial detachment (PED) at baseline was associated with significantly higher recurrence risk (74\% vs. 48\%).

Despite recurrence, vision could generally be recovered with prompt retreatment, suggesting that careful monitoring after discontinuation may be a viable approach for selected patients. However, the high recurrence rate underscores nAMD's chronic nature and the need for extended monitoring even after apparent disease stability \citep{aslanisRecurrentNeovascularAgeRelated2022}.

\subsection{Comparative Effectiveness}

When comparing anti-VEGF agents, the CATT study found bevacizumab to be non-inferior to ranibizumab in monthly dosing regimens \citep{RanibizumabBevacizumabNeovascular, martinRanibizumabBevacizumabTreatment2012}. Real-world comparisons among aflibercept, ranibizumab, and bevacizumab have likewise shown similar visual outcomes despite different molecular characteristics and theoretical advantages \citep{ciullaVisualAcuityOutcomes2020}.

\subsection{Conclusion}

The literature on aflibercept treatment for nAMD demonstrates robust efficacy in clinical trials and reasonable effectiveness in real-world settings, though with notable differences in outcome magnitude. Treatment protocols have evolved from fixed monthly or bimonthly regimens toward individualized T\&E approaches that balance treatment burden and efficacy. Future research should focus on optimizing patient selection for different treatment strategies and establishing reliable biomarkers for disease activity to guide individualized treatment decisions.

